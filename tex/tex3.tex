\documentclass[uplatex, dvipdfmx]{jsarticle}

\include{begin}

\section{『独習PHP』での記述の変化}

『独習PHP』という本では、以下のように版によって記述が異なる。

「独習PHP・第2版」(PHP5.3.2)
\begin{tcolorbox}
default\_charset = "UTF-8" \\
mbstring.language = Japanese \\
mbstring.internal\_encoding = UTF-8 \\
mbstring.encoding\_translation = Off \\
mbstring.http\_input = auto \\
mbstring.http\_output = pass \\
mbstring.detect\_order = UTF-8, SJIS, EUC-JP, JIS, ASCII \\
mbstring.substitute\_character = none 
\end{tcolorbox}

「独習PHP・第3版」(PHP7.0.1)

\begin{tcolorbox}
default\_charset = "UTF-8" とするだけで、mbstringについては言及なし。
\end{tcolorbox}

\rightline{(参考)『独習PHP・第3版』山田祥寛・著 \ 翔泳社 \ 2016年4月8日 \ 初版第1刷}

ちなみに、php.iniについての設定は、以下のようにすることを同書では勧めている。

\begin{tcolorbox}
 output\_buffering = 4096 \\
 error\_reporting = E\_ALL \& \textasciitilde E\_NOTICE \\
 default\_charset = ``UTF-8'' \\
 date.timezone = Asia/Tokyo \\
 session.use\_only\_cookies = On
\end{tcolorbox}




\include{end}

%% 修正時刻: Thu Feb 10 07:24:47 2022
