\documentclass[uplatex, dvipdfmx]{jsarticle}


\usepackage{tcolorbox}
\usepackage{color}
\usepackage{listings, plistings}

%% ノート/latexメモ
%% http://pepper.is.sci.toho-u.ac.jp/pepper/index.php?%A5%CE%A1%BC%A5%C8%2Flatex%A5%E1%A5%E2

% Java
\lstset{% 
  frame=single,
  backgroundcolor={\color[gray]{.9}},
  stringstyle={\ttfamily \color[rgb]{0,0,1}},
  commentstyle={\itshape \color[cmyk]{1,0,1,0}},
  identifierstyle={\ttfamily}, 
  keywordstyle={\ttfamily \color[cmyk]{0,1,0,0}},
  basicstyle={\ttfamily},
  breaklines=true,
  xleftmargin=0zw,
  xrightmargin=0zw,
  framerule=.2pt,
  columns=[l]{fullflexible},
  numbers=left,
  stepnumber=1,
  numberstyle={\scriptsize},
  numbersep=1em,
  language={Java},
  lineskip=-0.5zw,
  morecomment={[s][{\color[cmyk]{1,0,0,0}}]{/**}{*/}},
  keepspaces=true,         % 空白の連続をそのままで
  showstringspaces=false,  % 空白字をOFF
}
%\usepackage[dvipdfmx]{graphicx}
\usepackage{url}
\usepackage[dvipdfmx]{hyperref}
\usepackage{amsmath, amssymb}
\usepackage{itembkbx}
\usepackage{eclbkbox}	% required for `\breakbox' (yatex added)
\usepackage{enumerate}
\usepackage[default]{cantarell}
\usepackage[T1]{fontenc}
\fboxrule=0.5pt
\parindent=1em

\makeatletter
\def\verbatim@font{\normalfont
\let\do\do@noligs
\verbatim@nolig@list}
\makeatother

\begin{document}

%\anaumeと入力すると穴埋め解答欄が作れるようにしてる。\anaumesmallで小さめの穴埋めになる。
\newcounter{mycounter} % カウンターを作る
\setcounter{mycounter}{0} % カウンターを初期化
\newcommand{\anaume}[1][]{\refstepcounter{mycounter}{#1}{\boxed{\phantom{aa}\textnormal{\themycounter}\phantom{aa}}}} %穴埋め問題の空欄作ってる。
\newcommand{\anaumesmall}[1][]{\refstepcounter{mycounter}{#1}{\boxed{\tiny{\phantom{a}\themycounter \phantom{a}}}}}%小さい版作ってる。色々改造できる。

%% 修正時刻: Sat Oct 30 09:03:09 2021


\section{現状}

\subsection{php7入門ノートでの記述}


\subsection{PHPマニュアルでの記述}

\vspace{3mm}
\begin{tabular}{|l|l|l|} \hline
 mbstring.language & ``neutral'' &  \\
 mbstring.detect\_order & NULL & \\
 mbstring.http\_input & ``pass'' & 非推奨 \\
 mbstring.http\_output & ``pass'' &  非推奨 \\
 mbstring.internal\_encoding & NULL & 非推奨 \\
 mbstring.substitute\_character & NULL &  \\
 mbstring.func\_overload & ``0'' &  非推奨/PHP8で削除 \\
 mbstring.encoding\_translation & ``0'' & \\
 mbstring.http\_output\_conv\_mimetypes& (略) & \\
 mbstring.strict\_detection & ``0'' & \\ \hline
\end{tabular}
\vspace{3mm}

\href{https://www.php.net/manual/ja/mbstring.configuration.php}{実行時設定 PHPマニュアル}

\subsection{デフォルトでの状況}

以下の1箇所のみ記述を加えた状態で、あとはコメントのままだと、どうなるか?

\vspace{3mm}
\begin{tabular}{|l|} \hline
704 date.timezone = Asia/Tokyo \\

[mbstring] \\
 ;mbstring.language = Japanese \\
 ;mbstring.internal\_encoding = EUC-JP \\
 ;mbstring.http\_input = auto  \\
 ;mbstring.http\_output = SJIS \\
 ;mbstring.encoding\_translation = Off \\
 ;mbstring.detect\_order = auto \\
 ;mbstring.substitute\_character = none; \\
 ;mbstring.func\_overload = 0 \\
 ;mbstring.strict\_detection = Off \\ \hline
\end{tabular}
\vspace{3mm}

以下は、print\_r(mb\_get\_info()); の出力例

\begin{verbatim}
-------------------------------------------------------------------    
      Array
(
    [internal_encoding] => UTF-8
    [http_output] => UTF-8
    [http_output_conv_mimetypes] => ^(text/|application/xhtml\+xml)
    [func_overload] => 0
    [func_overload_list] => no overload
    [mail_charset] => UTF-8
    [mail_header_encoding] => BASE64
    [mail_body_encoding] => BASE64
    [illegal_chars] => 0
    [encoding_translation] => Off
    [language] => neutral
    [detect_order] => Array
        (
            [0] => ASCII
            [1] => UTF-8
        )

    [substitute_character] => 63
    [strict_detection] => Off
)
-------------------------------------------------------------------    
\end{verbatim}

ちなみに \textsf{default\_charset} は 初期値で \textsf{UTF-8} になっている。

\subsection{mbstring.language = Japanese を指定する}

コメントを削除して
\textsf{mbstring.language = Japanese} を指定した場合は、
デフォルトの場合から変化したのは、以下の箇所。

\begin{verbatim}
-------------------------------------------------------------------    
[mail_charset] => ISO-2022-JP
[mail_header_encoding] => BASE64
[mail_body_encoding] => 7bit
[language] => Japanese
[detect_order] => Array
    (
        [0] => ASCII
        [1] => JIS
        [2] => UTF-8
        [3] => EUC-JP
        [4] => SJIS
    )
-------------------------------------------------------------------    
\end{verbatim}

これらの設定は、mb\_send\_mail() を使うときに必要となる。

デフォルトのままで apache を起動し、あとから
\fbox{mb\_language('Japanese')} として
設定を変更させた場合は、以下のようになる。

\begin{verbatim}
-------------------------------------------------------------------    
[mail_charset] => ISO-2022-JP
[mail_body_encoding] => 7bit
[language] => Japanese
-------------------------------------------------------------------    
\end{verbatim}

\textsf{detect\_order} には反映されないので、\\
\fbox{mb\_detect\_order("ASCII,JIS,UTF-8,EUC-JP,SJIS");} \\
という関数を実行させる必要があるかもしれない。

ただ、\textsf{PHPMailer} を使って SMTP送信をするのには、mb\_send\_mail() は
必要ないので、mbstring.language = Japanese の設定は不要である。

\section{結論}

あとは初期値のままでかまわないというのであるなら、php.ini の設定箇所は、

\begin{tcolorbox}
date.timezone = Asia/Tokyo
\end{tcolorbox}

上の1箇所でいいということになる。

mbstring関連以外も含めると、以下のようになる。

\begin{tcolorbox}
307 memory\_limit = 512M \\
375 display\_errors = on \\
494 post\_max\_size = 512M \\
518 default\_charset = "UTF-8" \\
607 upload\_max\_filesize = 512M \\
705 date.timezone = Asia/Tokyo
\end{tcolorbox}

512M は、WordPressでバックアップを復元することを考慮した。




\end{document}

%% 修正時刻: Sat May  2 15:10:04 2020


%% 修正時刻: Wed Feb  9 21:39:51 2022
