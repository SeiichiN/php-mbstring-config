\documentclass[uplatex, dvipdfmx]{jsarticle}


\usepackage{tcolorbox}
\usepackage{color}
\usepackage{listings, plistings}

%% ノート/latexメモ
%% http://pepper.is.sci.toho-u.ac.jp/pepper/index.php?%A5%CE%A1%BC%A5%C8%2Flatex%A5%E1%A5%E2

% Java
\lstset{% 
  frame=single,
  backgroundcolor={\color[gray]{.9}},
  stringstyle={\ttfamily \color[rgb]{0,0,1}},
  commentstyle={\itshape \color[cmyk]{1,0,1,0}},
  identifierstyle={\ttfamily}, 
  keywordstyle={\ttfamily \color[cmyk]{0,1,0,0}},
  basicstyle={\ttfamily},
  breaklines=true,
  xleftmargin=0zw,
  xrightmargin=0zw,
  framerule=.2pt,
  columns=[l]{fullflexible},
  numbers=left,
  stepnumber=1,
  numberstyle={\scriptsize},
  numbersep=1em,
  language={Java},
  lineskip=-0.5zw,
  morecomment={[s][{\color[cmyk]{1,0,0,0}}]{/**}{*/}},
  keepspaces=true,         % 空白の連続をそのままで
  showstringspaces=false,  % 空白字をOFF
}
%\usepackage[dvipdfmx]{graphicx}
\usepackage{url}
\usepackage[dvipdfmx]{hyperref}
\usepackage{amsmath, amssymb}
\usepackage{itembkbx}
\usepackage{eclbkbox}	% required for `\breakbox' (yatex added)
\usepackage{enumerate}
\usepackage[default]{cantarell}
\usepackage[T1]{fontenc}
\fboxrule=0.5pt
\parindent=1em

\makeatletter
\def\verbatim@font{\normalfont
\let\do\do@noligs
\verbatim@nolig@list}
\makeatother

\begin{document}

%\anaumeと入力すると穴埋め解答欄が作れるようにしてる。\anaumesmallで小さめの穴埋めになる。
\newcounter{mycounter} % カウンターを作る
\setcounter{mycounter}{0} % カウンターを初期化
\newcommand{\anaume}[1][]{\refstepcounter{mycounter}{#1}{\boxed{\phantom{aa}\textnormal{\themycounter}\phantom{aa}}}} %穴埋め問題の空欄作ってる。
\newcommand{\anaumesmall}[1][]{\refstepcounter{mycounter}{#1}{\boxed{\tiny{\phantom{a}\themycounter \phantom{a}}}}}%小さい版作ってる。色々改造できる。

%% 修正時刻: Sat Oct 30 09:03:09 2021


\section{さまざなサイトでの設定例}

\subsection{PHP入門}

\href{https://webkaru.net/php/mbstring-php-ini/}{日本語環境の設定 - mbstring(PHP入門)}

(記事執筆時の日時は不明)    

\begin{tcolorbox}
\textgt{\Large mbstringの設定} \\
 
[mbstring] \\
mbstring.language = Japanese ← コメント(;)をはずします。\\
mbstring.internal\_encoding = UTF-8 ← コメントをはずし、内部エンコーディングをUTF-8へ。 \\
mbstring.http\_input = auto ← コメント(;)をはずします。 \\
mbstring.http\_output = UTF-8 ← コメントをはずし、エンコーディングをUTF-8へ。 \\
mbstring.encoding\_translation = On ← コメントをはずし、Onに。\\
mbstring.detect\_order = auto ← コメント(;)をはずします。
\end{tcolorbox}

「PHPマニュアル」についての言及はない。


\subsection{TechAcademy}

\href{https://techacademy.jp/magazine/39850}
{PHPでmbstringを設定して日本語環境に対応する方法を現役エンジニアが解説【初心者向け】(TechAcademy)}

(2019/6/30)

\begin{tcolorbox}
次にmbstringの設定を行います。

php.iniで「mbstring.internal\_encoding」を検索します。mbstring.internal\_encoding = 文字コード で設定することで、マルチバイト文字列関数でエンコードを行う際にここで設定された文字コードがデフォルト文字コードとなります。

下記の例では、文字コードを「UTF-8」に指定しています。

php.iniで「mbstring.language」を検索します。mbstring.language = 言語 で設定することで、mbstringで使用するデフォルトの言語を指定します。

下記の例では、デフォルトに「Japanese」を設定することで日本語に設定しています。

php.iniで「mbstring.detect\_order」を検索します。mbstring.detect\_order = 文字コード で設定することで、文字コードの自動判別を行う時にどの文字コードから順に確認していくのかを指定します。

最後にphp.iniファイルを保存し、設定は完了です。
\end{tcolorbox}


\subsection{Let'sプログラミング}

\href{https://www.javadrive.jp/php/install/index5.html#section5}
{php.iniファイルの作成と初期設定(Let'sプログラミング)}

(TATSUO IKURA)

(原稿執筆時の日時は不明)

\begin{tcolorbox}
; 次の 3 項目は PHP 8 で非推奨となりました \\
;mbstring.internal\_encoding = \\
;mbstring.http\_input = \\
;mbstring.http\_output = \\

[mbstring] \\
mbstring.language = Japanese \\
mbstring.encoding\_translation = Off \\
mbstring.detect\_order = UTF-8,SJIS,EUC-JP,JIS,ASCII \\
mbstring.substitute\_character = none \\
mbstring.strict\_detection = Off
\end{tcolorbox}

「PHP8で非推奨となりました」とあるが、実際は、PHP5.6あたりで非推奨に
なっている。

また、同じ 「TATSUO IKURA」の名前で書かれている次のサイトでは、
全く違う記述になっている。

\subsection{PHPBook}

\href{http://phpbook.s25.xrea.com/install/phpini/index5.html#section2}
{日本語利用の為の設定(PHPBook)}

(TATSUO IKURA)

(原稿執筆時の日時は不明)

\begin{tcolorbox}

[mbstring] \\
mbstring.language = Japanese \\
mbstring.internal\_encoding = UTF-8 \\
mbstring.http\_input = pass \\
mbstring.http\_output = pass \\
mbstring.encoding\_translation = Off \\
mbstring.detect\_order = UTF-8,SJIS,EUC-JP,JIS,ASCII \\
mbstring.substitute\_character = none \\
mbstring.func\_overload = 0 \\
mbstring.strict\_detection = Off \\
;mbstring.http\_output\_conv\_mimetype= \\
\end{tcolorbox}







\end{document}

%% 修正時刻: Sat May  2 15:10:04 2020


%% 修正時刻: Thu Feb 10 06:41:44 2022
